\section{Introduzione generale}


\subsection{Breve descrizione dello svolgimento degli esercizi}
Per ogni esercizio suddividiamo la sua descrizione in 4 parti fondamentali:

\begin{itemize}
    \item \textbf{Spiegazione teorica del problema}: in questa sezione viene fornita una descrizione formale del problema da risolvere, basata sugli assunti teorici affrontati nel corso e presenti nel libro di testo di Algoritmi e Strutture Dati [1].
    \item \textbf{Documentazione del codice}: qui si illustra nel dettaglio l'implementazione dell'esercizio, evidenziando le scelte progettuali, la struttura del codice e le eventuali ottimizzazioni adottate.
    \item \textbf{Descrizione degli esperimenti condotti}: a partire dal codice sviluppato, si eseguono una serie di misurazioni volte a verificare sperimentalmente le ipotesi teoriche formulate in precedenza.
    \item \textbf{Analisi dei risultati sperimentali}: si analizzano criticamente i dati raccolti durante la fase sperimentale, confrontandoli con le previsioni teoriche, infine si formulano osservazioni e considerazioni conclusive.
\end{itemize}

\subsection{Specifiche della piattaforma di test}
La piattaforma di test sarà la stessa per ogni esercizio che vedremo. L'hardware del computer usato per testare questi esercizi è il seguente:

\begin{itemize}
    \item \textbf{CPU}: 1,8 GHz Intel Core i5 dual-core
    \item \textbf{RAM}: 8 GB 1600 MHz DDR3
    \item \textbf{SSD}: APPLE SSD SM0128G Media 121GB

\end{itemize}

Il linguaggio di programmazione utilizzato sarà Python e la piattaforma in cui il codice è stato scritto e testato è l'IDE \textbf{PyCharm 2025.1.1.1}. La stesura di questo testo è avvenuta tramite l'utilizzo dell'editor online \textbf{Overleaf}.
